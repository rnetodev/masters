\resumo

Com a grande quantidade dados produzidos e coletados diariamente por diferentes sistemas, técnicas de aprendizado de máquina foram propostas com o intuito de auxiliar no processo de extração automática de informações. Dentre essas técnicas, pode-se destacar os algoritmos de agrupamento que buscam encontrar padrões e estruturas implícitas em conjuntos de dados sem qualquer conhecimento fornecido à priori. Neste trabalho de mestrado, busca-se desenvolver uma nova abordagem de agrupamento para dados que possuem uma dependência temporal entre suas observações, comumente chamados de séries temporais. A principal diferença dessa abordagem em relação aos trabalhos existentes na literatura baseia-se na hipótese de que dados coletados do mundo real possuem influências estocásticas e determinísticas que, se não forem individualmente analisadas, podem afetar o resultado do agrupamento. Neste sentido, a abordagem proposta realiza uma etapa de decomposição das séries temporais em componentes estocásticos e determinísticos. Em seguida, realiza o agrupamento dos dados analisando de maneira individual a similaridade entre cada componente. Experimentos preliminares foram realizados sobre séries temporais sintéticas, formadas a partir da combinação entre componentes estocásticos e determinísticos. A partir da decomposição das séries sintéticas, foi possível observar que medidas de distância largamente utilizadas na literatura apresentaram melhores resultados. Ao afinal deste trabalho de mestrado, espera-se obter um melhor resultado de agrupamento utilizando a abordagem proposta sobre dados reais.

% Palavras-chave do resumo em Portugues
\begin{keywords}
Agrupamento, Séries Temporais, Decomposição, Estocasticidade, Determinismo. 
\end{keywords}