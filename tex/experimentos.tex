\xchapter{Experimentos Iniciais}{}\label{experimentos}

\section{Considerações Iniciais}

Este capítulo apresenta um conjunto de experimentos preliminares realizados sobre 40 séries temporais sintéticas, cujo principal objetivo foi demonstrar a viabilidade da proposta de mestrado. Neste sentido, resultados de análises empíricas justificaram a importância de calcular a distância entre séries temporais considerando, individualmente, as influências estocásticas e determinísticas de seus componentes.

Para execução dos experimentos, a distância entre pares de séries temporais foi calculada da seguinte forma: i) aplicação direta do cálculo da distância entre séries temporais ruidosas; ii) decomposição dos componentes estocásticos e determinísticos de cada série e, posterior, cálculo da distância apenas entre os componentes determinísticos.

Nestes experimentos, o componente estocástico foi descartado como um ruído que não adiciona qualquer nova informação relevante aos dados. Entretanto, no projeto proposto neste trabalho, a estocasticidade presente nas séries será analisada de maneira semelhante ao componente determinístico.

A decomposição das séries em componentes estocásticos e determinísticos foi realizada utilizando as etapas propostas por \citeonline{Araujo2013, Araujo2015}. A próxima seção apresenta como cada série sintética foi produzida para condução dos experimentos.

\section{Configuração dos experimentos}\label{conf:exp}

As séries temporais sintéticas utilizadas nos experimentos foram criadas combinando os seguintes componentes: 

\begin{enumerate}
\item Determinístico:observações geradas a partir da função seno e cosseno com frequência angular igual a $\pi$ e $2\pi$;
\item Estocástico: observações geradas a partir de um ruído branco com média igual a $0$ e desvio padrão entre $0,1$ e $1,0$;
\item Tendência: observações representando uma tendência linear positiva nos dados.
\end{enumerate}

Estes componentes permitiram gerar grupos de séries temporais da seguinte forma:

\begin{itemize}
\item TIPO 1: 10 séries cossenoides com ruído branco.
\item TIPO 2: 10 séries cossenoide com ruído branco e tendência.
\item TIPO 3: 10 séries senoide com ruído branco.
\item TIPO 4: 10 series senoide com ruído branco e tendência.
\end{itemize}

Todas as séries temporais foram criadas com $3000$ observações e detalhes sobre a adição dos ruídos são apresentados na Tabela \ref{series}.

\begin{table}[!ht]
\centering
\caption{Grupos de séries temporais sintéticas utilizadas nos experimentos}
\label{series}
\begin{tabular}{lllll}
\hline
Desvio Padrão & TIPO 1     & TIPO 2     & TIPO 3     & TIPO 4     \\
\hline
\hline
0.1           & Série 1.1  & Série 2.1  & Série 3.1  & Série 4.1  \\
0.2           & Série 1.2  & Série 2.2  & Série 3.2  & Série 4.2  \\
0.3           & Série 1.3  & Série 2.3  & Série 3.3  & Série 4.3  \\
0.4           & Série 1.4  & Série 2.4  & Série 3.4  & Série 4.4  \\
0.5           & Série 1.5  & Série 2.5  & Série 3.5  & Série 4.5  \\
0.6           & Série 1.6  & Série 2.6  & Série 3.6  & Série 4.6  \\
0.7           & Série 1.7  & Série 2.7  & Série 3.7  & Série 4.7  \\
0.8           & Série 1.8  & Série 2.8  & Série 3.8  & Série 4.8  \\
0.9           & Série 1.9  & Série 2.9  & Série 3.9  & Série 4.9  \\
1.0           & Série 1.10 & Série 2.10 & Série 3.10 & Série 4.10 \\
\hline
\end{tabular}
\end{table}

Após a criação das séries temporais, suas observações foram normalizadas conforme a Equação \ref{norma}, sendo que $\hat{X}(t)$ representa a série temporal $X(t)$ normalizada, e $max(\cdot)$ e $min(\cdot)$ representam o maior e o menor valor de $X(t)$, respectivamente.


\begin{equation}\label{norma}
\hat{X}(T) =\frac{X(t)-min(X(t))}{max(X(t))-min(X(t))}
 \end{equation}

Por fim, a decomposição dos componentes estocásticos e determinísticos das 40 séries normalizadas foi executada utilizando o método EMD (\emph{Empirical Mode Decomposition}), conforme descrito em \citeonline{Araujo2013, Araujo2015}. Inicialmente, cada série foi decomposta em um conjunto de IMFs (\emph{Intrinsic Mode Function}), as quais foram posteriormente combinadas para formar os componentes estocásticos e determinísticos. As IMFs (\emph{Intrinsic Mode Function}) obtidas pelo método EMD (\emph{Empirical Mode Decomposition}) para cada série temporal analisada estão listadas no Apêndice \ref{apendice1}.

\section{Análise de Similaridade/Distância}

Esta seção apresenta um conjunto de experimentos realizados com o objetivo de validar esta proposta de mestrado. Embora os experimentos ainda não sejam suficientes para comprovar a hipótese deste trabalho, a análise empírica apresentada neste exame de qualificação é importante para evidenciar a relevância de analisar, de maneira, individual as influências dos componentes estocásticos e determinísticos no cálculo da similaridade e/ou da distância entre séries temporais.

Para execução dos experimentos, foram selecionadas 6 medidas de distância e similaridade listadas no Capítulo \ref{revisao}: i) DTW; ii) Euclidiana; iii) Manhattan; iv) Minkowski; v) CID; e vi) Cross-Correlation.

O primeiro experimento foi realizado executando usando a DTW para calcular a distância entre duas séries puramente determinísticas criadas usando uma função seno e uma função cosseno. Neste experimento, a distância entre as duas séries foi igual a $90.41829$. Em seguida, calculou-se a distância entre pares de séries do Tipo 1 e do Tipo 3. Conforme detalhado na Seção \ref{conf:exp}, essas séries foram criadas a partir da adição de ruído em séries geradas a parir da função seno e cosseno. O resultado desta análise pode ser visto na Tabela \ref{dtw}.

% latex table generated in R 3.3.3 by xtable 1.8-2 package
% Wed Apr 26 16:25:43 2017
\begin{table}[!ht]
\centering
\caption{Cálculo da DTW entre séries do Tipo 1 e Tipo 3 (ruído aditivo).}
\begin{tabular}{ll}
  \hline
 Séries & Distância  \\
  \hline
  \hline
(Série 1.1, Série 3.1) & 297.34 \\
(Série 1.2, Série 3.2)& 523.89 \\
(Série 1.3, Série 3.3) & 762.22 \\
(Série 1.4, Série 3.4) & 1013.72 \\
(Série 1.5, Série 3.5) & 1262.14\\
(Série 1.6, Série 3.6) & 1497.14\\
(Série 1.7, Série 3.7) & 1729.46 \\
(Série 1.8, Série 3.8) & 1990.24\\
(Série 1.9, Série 3.9) & 2270.85\\
(Série 1.10, Série 3.10) & 2459.25 \\
\hline
\end{tabular}
\label{dtw}
\end{table}

Como pode ser observado na Tabela \ref{dtw}, o valor da distância entre as séries tende a aumentar à medida que suas observações são mais influenciadas pelo componente estocástico. O mesmo comportamento é observado quando a tendência é adicionada à série, além do componente estocástico, conforme apresentado na Tabela \ref{dtw2}. Neste experimento, pôde-se notar que a inclusão da tendência não alterou significativamente o comportamento da DTW. Aplicações de um teste de homogeneidade de variância (F-test de Fisher) e do t-student retornaram $p$ valores iguais a $0,8841$ e $0,7002$, respectivamente, evidenciando que as distâncias nos dois experimentos são similares.

\begin{table}[!ht]
\centering
\caption{Cálculo da DTW entre séries do Tipo 2 e Tipo 4 (ruído aditivo e tendência).}
\begin{tabular}{ll}
 \hline
Séries & Distância \\
 \hline
 \hline
(Série 2.1, Série 4.1) & 437.62 \\ 
(Série 2.2, Série 4.2) & 691.64 \\ 
(Série 2.3, Série 4.3) & 944.55 \\ 
(Série 2.4, Série 4.4) & 1159.95 \\
(Série 2.5, Série 4.5) & 1418.49 \\ 
(Série 2.6, Série 4.6) & 1634.87 \\ 
(Série 2.7, Série 4.7) & 1847.99 \\ 
Série 2.8, Série 4.8) & 2082.55 \\ 
Série 2.9, Série 4.9) & 2292.85 \\ 
(Série 2.10, Série 4.10) & 2556.13 \\
  \hline
\end{tabular}
\label{dtw2}
\end{table}

Em seguida, aplicou-se o método de decomposição de séries em componentes estocásticos e determinísticos sobre as séries do Tipo 1 e do Tipo 3, cujas distâncias foram apresentadas na Tabela \ref{dtw}. Nos experimentos com a decomposição, o componente estocástico foi desconsiderado na análise. Por outro lado, as distâncias entre os componentes determinísticos foram calculadas usando DTW. Conforme resultados apresentados na Tabela \ref{dtwcompor}, as distâncias entre as séries ruidosas reduziram significativamente com a aplicação da decomposição. Aplicando novamente os testes de hipótese de homogeneidade de variância (F-test de Fisher) e t-student, obteve-se $p$ valores iguais a $3,814 \cdot 10^{-08}$ e $0.0005$, respectivamente, evidenciando que as médias das distâncias nos dois experimentos não são similares.

% latex table generated in R 3.3.3 by xtable 1.8-2 package
% Wed Apr 26 15:50:53 2017
\begin{table}[!ht]
\centering
\caption{Cálculo da DTW entre séries do Tipo 1 e Tipo 3 com decomposição.}
\begin{tabular}{ll}
  \hline
  Séries (Componentes Determinísticos) & Distância \\
  \hline
  \hline
(Série 1.1, Série 3.1) & 96.43 \\ 
(Série 1.2, Série 3.2) & 91.03 \\
(Série 1.3, Série 3.3) & 108.73\\ 
(Série 1.4, Série 3.4) & 219.18  \\ 
(Série 1.5, Série 3.5) & 156.10 \\
(Série 1.6, Série 3.6) & 230.97 \\
(Série 1.7, Série 3.7) & 184.60 \\
(Série 1.8, Série 3.8)& 59.28 \\ 
(Série 1.9, Série 3.9) & 185.75 \\
(Série 1.10, Série 3.10) &  239.52 \\ 
   \hline
\end{tabular}
\label{dtwcompor}
\end{table}

Por fim, as mesmas etapas de decomposição foram executadas sobre as séries com ruído aditivo e tendência (séries do Tipo 2 e 4). Em seguida, analisou-se a distância entre os componentes determinísticos. Assim como o resultado anterior, foi possível observar nesta análise que os valores obtidos para distância não tiveram comportamento explosivo à medida que a influência do componente estocástico aumentou. 




\begin{table}[!ht]
\centering
\caption{Cálculo da DTW entre séries do Tipo 2 e Tipo 4 com decomposição.}
\begin{tabular}{ll}
 \hline
   Séries (Componentes Determinísticos) & Distância \\
 \hline
(Série 2.1, Série 4.1) & 246.68 \\
(Série 2.2, Série 4.2) & 283.72 \\
(Série 2.3, Série 4.3) & 359.91 \\ 
(Série 2.4, Série 4.4) & 351.89 \\ 
(Série 2.5, Série 4.5) & 357.99 \\
(Série 2.6, Série 4.6) & 326.95 \\ 
(Série 2.7, Série 4.7) & 54.89 \\
(Série 2.8, Série 4.8) & 126.59 \\
(Série 2.9, Série 4.9) & 220.07  \\ 
(Série 2.10, Série 4.10) & 197.73  \\
  \hline
\end{tabular}
\label{dtwcompor2}
\end{table}


Aplicando de maneira similar os testes de hipótese de homogeneidade de variância (F-test de Fisher) e t-student sobre as distâncias sem decomposição e as distâncias entre os componentes determinístico, obteve-se $p$ valores iguais a $4,005 \cdot 10^{-06}$ e $0,0002$, respectivamente, evidenciando que as médias das distâncias entre os dois experimentos não são similares.

Os experimentos realizados anteriormente foram repetidos alterando apenas a medida de distância utilizada. Neste novo conjunto de experimentos, a DTW foi substituída pela distância Euclidiana. A Tabela \ref{euclidiana} apresenta os resultados das distâncias calculadas diretamente sobre as séries do Tipo 1 e 3, e sobre seus componentes determinísticos após a etapa de decomposição. É importante destacar que a distância euclidiana calculada sobre as séries puramente determinísticas produzidas a partir das funções seno e cosseno foi igual a $27,38614$.

\begin{table}[!ht]
\centering
\caption{Cálculo da medida Euclidiana entre séries do Tipo 1 e Tipo 3 sem decomposição e com decomposição.}
\begin{tabular}{lll}
  \hline
 Séries & Euclidiana (sem decomposição) & Euclidiana (decomposição) \\
  \hline
  \hline
(Série 1.1, Série 3.1) & 28.56 & 27.24 \\ 
(Série 1.2, Série 3.2) & 32.04 & 27.75\\
(Série 1.3, Série 3.3) & 36.24 & 28.23\\ 
(Série 1.4, Série 3.4) & 41.03 & 27.51\\
(Série 1.5, Série 3.5) & 48.14 & 24.87 \\
(Série 1.6, Série 3.6) & 53.48 & 29.18\\
(Série 1.7, Série 3.7) & 59.69 & 27.01\\ 
(Série 1.8, Série 3.8) & 66.93 & 27.65\\ 
(Série 1.9, Série 3.9) & 74.61 & 26.84\\ 
(Série 1.10, Série 3.10) & 82.13 & 26.09\\ 
   \hline
\end{tabular}
\label{euclidiana}
\end{table}

Como pode ser observado nessa tabela, os resultados com a distância euclidiana foram semelhantes aos resultados obtidos com a DTW. Os resultados obtidos com a decomposição foram similares à distância obtida com as séries puramente determinísticas. 

Aplicando os testes F-test de Fisher e t-student sobre as distâncias sem decomposição e as distâncias entre os componentes determinísticos, obteve-se $p$ valores iguais a $2,291 \cdot 10^{-09}$ e $0.001$, respectivamente, evidenciando que as médias das distâncias entre os dois experimentos não são similares. Além disso, pode-se observar que os valores obtidos com a decomposição foram próximos dos valores esperados.

Na Tabela \ref{euclidiana2}, os experimentos foram repetidos para as séries do Tipo 2 e 4, i. e., séries cujas observações são compostas por um ruído aditivo quanto pela presença de uma tendência positiva. Os resultados obtidos foram similares aos resultados apresentados na Tabela \ref{euclidiana}. Além disso, os $p$ valores obtidos para o F-test de Fisher e o t-student foram, respectivamente,  $3,763 \cdot 10^{-07}$ e $0.001$, enfatizando as diferenças entre os resultados obtidos.

\begin{table}[!ht]
\centering
\caption{Cálculo da medida Euclidiana entre séries do Tipo 2 e Tipo 4 sem decomposição e com decomposição.}
\begin{tabular}{lll}
 \hline
Séries & Euclidiana (sem decomposição) & Euclidiana (decomposição) \\
 \hline
 \hline
(Série 2.1, Série 4.1) & 28.54 & 25.14\\ 
(Série 2.2, Série 4.2) & 31.48 & 26.55\\ 
(Série 2.3, Série 4.3) & 35.58 & 25.72\\
(Série 2.4, Série 4.4) & 41.61 & 27.21\\ 
(Série 2.5, Série 4.5) & 48.24 & 30.85\\
(Série 2.6, Série 4.6) & 54.74 & 28.60\\
(Série 2.7, Série 4.7) & 60.60 & 23.41\\ 
(Série 2.8, Série 4.8) & 68.74 & 27.41\\ 
(Série 2.9, Série 4.9) & 74.60 & 27.56\\
(Série 2.10, Série 4.10) & 82.33 & 29.18\\ 
  \hline
\end{tabular}
\label{euclidiana2}
\end{table}


O próximo conjunto de testes foi realizado utilizando a distância de Manhattan, cujo valor entre as duas séries puramente determinística foi igual a $1350,525$. A Tabela \ref{manhattan} apresenta as distâncias utilizando essa medida para séries do Tipo 1 e 3 calculadas antes e após a decomposição. Assim como nos experimentos anteriores, além da decomposição permitir obter valores próximos ao esperado (antes da adição de ruído), os testes de hipóteses confirmaram que existe uma diferença entre os resultados obtidos (F-test de Fisher igual a $7,876 \cdot 10^{-08}$  e o t-student igual a $0.002$).

\begin{table}[!ht]
\centering
\caption{Cálculo da distância Manhattan entre séries do Tipo 1 e Tipo 3 sem decomposição e com decomposição.}
\begin{tabular}{lll}
  \hline
Séries & Manhattan (sem decomposição) & Manhattan (decomposição) \\
  \hline
(Série 1.1, Série 3.1) & 1380.60 & 1341.64\\
(Série 1.2, Série 3.2) & 1497.11 & 1373.13\\ 
(Série 1.3, Série 3.3) &  1641.60 & 1394.78 \\ 
(Série 1.4, Série 3.4) & 1817.47 & 1326.44\\
(Série 1.5, Série 3.5) & 2115.14 & 1160.13\\
(Série 1.6, Série 3.6) & 2366.99 & 1401.63\\
(Série 1.7, Série 3.7) & 2602.16 & 1316.90\\ 
(Série 1.8, Série 3.8) & 2943.43 & 1312.15\\ 
(Série 1.9, Série 3.9) & 3256.53 & 1278.53\\ 
(Série 1.10, Série 3.10) &  3629.29 & 1242.96\\ 
   \hline
\end{tabular}
\label{manhattan}
\end{table}

Os resultados apresentados na Tabela \ref{manhattan2}, obtidos a partir da adição de tendência, foram similares aos resultados apresentados anteriormente. Neste experimento, os valores obtidos foram próximos do valor esperado e os $p$ valores do F-test de Fisher e do t-student foram iguais a $1.66 \cdot 10^{-06}$ e $0.002$, respectivamente.

\begin{table}[!ht]
\centering
\caption{Cálculo da distância Manhattan entre séries do Tipo 2 e Tipo 4 sem decomposição e com decomposição.}
\begin{tabular}{lll}
 \hline
Séries & Manhattan (sem decomposição) & Manhattan (decomposição) \\
 \hline
(Série 2.1, Série 4.1) & 1379.57 & 1217.93 \\ 
(Série 2.2, Série 4.2) & 1472.45 & 1299.72 \\
(Série 2.3, Série 4.3) & 1609.70 & 1228.13 \\ 
(Série 2.4, Série 4.4) & 1848.52 & 1308.81 \\ 
(Série 2.5, Série 4.5) &  2142.42 & 1429.08 \\ 
(Série 2.6, Série 4.6) & 2429.22 & 1370.57 \\
(Série 2.7, Série 4.7) & 2678.07 & 1063.83 \\
(Série 2.8, Série 4.8) & 3001.11 & 1320.86 \\ 
(Série 2.9, Série 4.9) & 3250.85 & 1328.37 \\ 
(Série 2.10, Série 4.10) & 3597.55 & 1390.20 \\
  \hline
\end{tabular}
\label{manhattan2}
\end{table}

Para os experimentos apresentados a seguir, as tabelas contendo as análise entre as séries do Tipo 1 e 3 e as séries do Tipo 2 e 4 foram concatenadas visando discutir seus resultados de maneira mais sucinta. Assim, analisando a Tabela \ref{Minkowski}, observou-se que a decomposição permitiu encontrar valores similares para a distância esperada de Minkowski, cujo valor era $7,663852$. Os valores do F-test de Fisher e do t-student foram iguais a $1,66 \cdot 10^{-06}$ e $0,002$, respectivamente. Em seguida, foram analisadas as séries ruidosas com adição de tendência. Os resultados obtidos foram próximos do valor esperado e os $p$ valores do F-test de Fisher e do t-student foram iguais a $2.316 \cdot 10^{-06}$ e $0.0008$, respectivamente.

\begin{table}[!ht]
\centering
\caption{Cálculo da distância Minkowski entre séries sem decomposição e com decomposição.}
\begin{tabular}{lll}
 \hline
Séries (1 e 3) & Minkowski (sem decomposição) & Minkowski (decomposição) \\
 \hline
 \hline
(Série 1.1, Série 3.1)  & 8.14 & 7.64\\ 
(Série 1.2, Série 3.2)  &  9.38 & 7.76 \\ 
(Série 1.3, Série 3.3)  & 10.84 & 7.90 \\ 
(Série 1.4, Série 3.4)  & 12.48 & 7.81 \\ 
(Série 1.5, Série 3.5)  & 14.73 & 7.21\\ 
(Série 1.6, Série 3.6)  & 16.30 & 8.33 \\
(Série 1.7, Série 3.7)  & 18.40 & 7.62 \\ 
(Série 1.8, Série 3.8)  & 20.53 & 7.93 \\ 
(Série 1.9, Série 3.9)  & 22.95 & 7.69 \\ 
(Série 1.10, Série 3.10)  & 25.10 & 7.53 \\ 
\hline
Séries (2 e 4) & Minkowski (sem decomposição) & Minkowski (decomposição) \\
\hline
\hline
(Série 2.1, Série 4.1) & 8.13  & 7.17\\ 
(Série 2.2, Série 4.2) & 9.22 & 7.47\\
(Série 2.3, Série 4.3) & 10.68  & 7.42 \\
(Série 2.4, Série 4.4) & 12.63 & 7.76 \\
(Série 2.5, Série 4.5) & 14.64 & 9.07 \\ 
(Série 2.6, Série 4.6) & 16.70 &  8.17 \\ 
(Série 2.7, Série 4.7) & 18.52 & 6.82 \\ 
(Série 2.8, Série 4.8) & 21.15 & 7.82 \\
(Série 2.9, Série 4.9) & 22.92 & 7.84 \\
(Série 2.10, Série 4.10) & 25.33 & 8.35 \\
 \hline
\end{tabular}
\label{Minkowski}
\end{table}

O último experimento utilizando medidas de distância foi realizado com a medida CID. A Tabela \ref{CID} apresenta os resultados obtidos e é possível observar que a decomposição permitiu encontrar valores similares para a distância esperada para as séries puramente determinística, cujo valor era $27,38615$. Os valores do F-test de Fisher e do t-student foram iguais a $6,164 \cdot 10^{-06}$ e $0,004$, respectivamente. Em seguida, foram analisadas as séries ruidosas com adição de tendência. Os resultados obtidos foram próximos do valor esperado e os $p$ valores do F-test de Fisher e do t-student foram iguais a $5,742 \cdot 10^{-05}$ e $0.004$, respectivamente.

\begin{table}[!ht]
\centering
\caption{Cálculo da distância CID entre séries sem decomposição e com decomposição.}
\begin{tabular}{lll}
 \hline
Séries (1 e 3) & CID (sem decomposição) & CID (decomposição) \\
 \hline
(Série 1.1, Série 3.1)  & 28.86 & 27.83 \\ 
(Série 1.2, Série 3.2)  &  33.01 & 27.92 \\
(Série 1.3, Série 3.3)  & 36.64 & 28.36 \\ 
(Série 1.4, Série 3.4)  &  41.07 & 34.30 \\ 
(Série 1.5, Série 3.5)  & 48.21 & 27.80  \\
(Série 1.6, Série 3.6)  & 54.35 & 35.18 \\
(Série 1.7, Série 3.7)  & 61.72 & 27.39 \\ 
(Série 1.8, Série 3.8)  & 67.01 & 28.80 \\
(Série 1.9, Série 3.9)  & 77.95 & 32.64 \\ 
(Série 1.10, Série 3.10)  & 83.45 & 28.54 \\
\hline
Séries (2 e 4) & CID (sem decomposição) & CID (decomposição) \\
\hline
\hline
(Série 2.1, Série 4.1) & 28.91  & 27.14\\ 
(Série 2.2, Série 4.2) & 31.82 &  28.05 \\ 
(Série 2.3, Série 4.3) & 36.73 & 29.05 \\
(Série 2.4, Série 4.4) & 42.53 & 35.78 \\ 
(Série 2.5, Série 4.5) & 48.82 & 33.50 \\
(Série 2.6, Série 4.6) & 56.56 & 36.49 \\
(Série 2.7, Série 4.7) & 61.01 & 24.30 \\ 
(Série 2.8, Série 4.8) & 69.87 & 28.68 \\
(Série 2.9, Série 4.9) & 75.93 & 28.45 \\
(Série 2.10, Série 4.10) & 85.16 & 29.56 \\ 
\hline
\end{tabular}
\label{CID}
\end{table}

O último experimento apresentado neste capítulo foi realizado com a medida de distância Cross-correlation. Esta medida é altamente sensível à qualquer variação nos dados. Por exemplo, a distância entre duas senoides idênticas é igual a $0$. Contudo, ao calcular a distância entre uma senoide e uma cossenoide, obtém-se um valor maior que $0$, uma vez que as observações entre as duas séries não estão alinhadas. Sendo assim, visando verificar que a adição de ruído e tendência afeta diretamente esta medida, analisou-se apenas as séries Tipo 1 e 2. Com a abordagem estudada neste projeto, espera-se que a decomposição permita correlacionar séries com comportamento determinístico semelhantes. Os resultados obtidos estão apresentados na Tabela~\ref{cross} e, de maneira similar aos experimentos anteriores, confirmam a viabilidade da proposta. 


\begin{table}[!ht]
\centering
\caption{Cálculo da distância Cross Correlation entre séries sem decomposição e com decomposição.}
\begin{tabular}{lll}
 \hline
Séries (1 e 2) & Cross (sem decomposição) & Cross (decomposição) \\
 \hline
(Série 1.1, Série 2.1)  & 0.24 & 0.03 \\ 
(Série 1.2, Série 2.2)  &  0.26  & 0.01 \\
(Série 1.3, Série 2.3)  &  0.30 & 0.01 \\ 
(Série 1.4, Série 2.4)  &  0.35 & 0.02 \\ 
(Série 1.5, Série 2.5)  & 0.42 & 0.03  \\
(Série 1.6, Série 2.6)  & 0.44 & 0.03 \\
(Série 1.7, Série 2.7)  & 0.50 & 0.03 \\ 
(Série 1.8, Série 2.8)  & 0.60 & 0.02 \\
(Série 1.9, Série 2.9)  & 0.59 & 0.03 \\ 
(Série 1.10, Série 2.10)  & 0.52 & 0.02 \\
\hline
\end{tabular}
\label{cross}
\end{table}

Por fim, é importante destacar que os resultados obtidos foram próximos do valor esperado ($0$) e os $p$ valores do F-test de Fisher e do t-student foram iguais a $3,045 \cdot 10^{-09}$ e $4,809 \cdot 10^{-06}$, respectivamente.

\section{Considerações Finais}

Nesta seção, foram apresentados resultados dos primeiros experimentos executados neste trabalho, os quais visavam verificar a importância da utilização da decomposição de séries temporais na utilização de métricas de similaridade/distância. 

Nos experimentos futuros, está previsto o estudo de outras medidas para análise do comportamento determinístico, como RQA-RP (\emph{Recurrence Quantification Analysis -- Recurrence plot}), e medidas para o comportamento estocástico. Dentre as medidas utilizadas para análise do comportamento estocástico, pode-se citar a transformação dos componentes no domínio de frequência (Transformada de Fourier) para posterior análise dos espectrogramas.

Por fim, em experimentos futuros, espera-se validar a hipótese desse trabalho com a aplicação de todas as etapas do agrupamento de séries temporais, incluindo a execução de algoritmos de agrupamento e a aplicação de técnicas de validação. Espera-se, ainda, realizar uma aplicação prática da abordagem proposta, não apenas em dados sintéticos, mas em um conjunto de dados coletados a partir de um sistema do mundo real.



%*****
 
%A similaridade entre as séries, antes de passar pelo processo de decomposição, diminui conforme o desvio padrão do ruído for aumentando. É possível observar tal comportamento neste experimento na DTW (Tabela \ref{dtw}, Tabela \ref{dtw2}), Euclidiana(Tabela \ref{euclidiana}, Tabela \ref{euclidiana}), Minkowski (Tabela \ref{Minkowski}, Tabela \ref{Minkowski2}) e CID(Tabela \ref{CID}, Tabela \ref{CID2}). No entanto, a \textit{Cross-Correlation} devido ao \textit{lag} [-1,1] não conseguiu diferenciar os resultados entre algumas séries com ruído e tendência, como é mostrado na Tabela \ref{CROSS2}. 

%O cálculo entre os componentes determinísticos com a distância CID, Euclidiana, Manhattan e Minkowski apresentaram um resultado aproximado ao resultado do cálculo das séries originais senoide e cossenoide. No entando, a DTW mostrou uma diferença para alguns resultados do cálculo entre os componentes mostrados na Tabela \ref{dtwcompor}  e Tabela \ref{dtwcompor2}, quando comparados ao resultado do cálculo entre a séries originais ( $90.41829$). É importante ressaltar que a natureza da decomposição gera um componente determinístico muito próximo ao original, muitas vezes não sendo igual, e como a DTW busca um alinhamento ótimo entre as séries, gera uma variação entre os resultados. 

%Diante dos resultados satisfatórios obtidos até o presente momento, nos próximos passos da pesquisa serão tratados os componentes estocásticos. Sendo considerado para o cálculo da similaridade entre os componentes estocásticos serem realizados a partir da análise no domínio de frequência, comparando espectrogramas obtidos com a transformada de Fourier~\cite{morettin2006}.